%% no need for  \DeclareGraphicsExtensions{.pdf,.eps}

\documentclass[12pt,letterpaper,english]{article}
\usepackage{times}
\usepackage[T1]{fontenc}
\IfFileExists{url.sty}{\usepackage{url}}
                      {\newcommand{\url}{\texttt}}

\usepackage{babel}
%\usepackage{noweb}
\usepackage{Rd}

\usepackage{Sweave}

%\VignetteIndexEntry{Performance Attribution from Bacon}
%\VignetteDepends{PerformanceAnalytics}
%\VignetteKeywords{returns, performance, risk, benchmark, portfolio}
%\VignettePackage{PerformanceAnalytics}

%\documentclass[a4paper]{article}
%\usepackage[noae]{Sweave}
%\usepackage{ucs}
%\usepackage[utf8x]{inputenc}
%\usepackage{amsmath, amsthm, latexsym}
%\usepackage[top=3cm, bottom=3cm, left=2.5cm]{geometry}
%\usepackage{graphicx}
%\usepackage{graphicx, verbatim}
%\usepackage{ucs}
%\usepackage[utf8x]{inputenc}
%\usepackage{amsmath, amsthm, latexsym}
%\usepackage{graphicx}

\title{Chekhlov Conditional Drawdown at Risk}
\author{R Project for Statistical Computing}

\begin{document}
\Sconcordance{concordance:ConditionalDrawdown.tex:ConditionalDrawdown.Rnw:%
1 45 1 1 5 1 4 21 1 1 2 1 0 2 1 17 0 1 2 3 1}


\maketitle


\begin{abstract}
A new one-parameter family of risk measures called Conditional Drawdown (CDD) has
been proposed. These measures of risk are functionals of the portfolio drawdown (underwater) curve considered in active portfolio management. For some value of $\hat{\alpha}$ the tolerance parameter, in the case of a single sample path, drawdown functional is defined as the mean of the worst (1 \(-\) $\hat{\alpha}$)100\% drawdowns. The CDD measure generalizes the notion of the drawdown functional to a multi-scenario case and can be considered as a generalization of deviation measure to a dynamic case. The CDD measure includes the Maximal Drawdown and Average Drawdown as its limiting cases.
\end{abstract}



\section{Background}

The model is focused on concept of drawdown measure which is in possession of all properties of a deviation measure,generalization of deviation measures to a dynamic case.Concept of risk profiling - Mixed Conditional Drawdown (generalization of CDD).Optimization techniques for CDD computation - reduction to linear programming (LP) problem. Portfolio optimization with constraint on Mixed CDD
The model develops concept of drawdown measure by generalizing the notion
of the CDD to the case of several sample paths for portfolio uncompounded rate
of return.


\section{Methodology}
For a given value of sequence ${\xi_k}$ where ${\xi}$ is a time series unit drawdown vector.The CV@R is formally defined as : 
\begin{equation}
CV@R_{\alpha}(\xi)=\frac{\pi_{\xi}(\zeta(\alpha))-\alpha}{1-\alpha}\zeta(\alpha) + \frac{ \sum_{\xi_k=1}^{} \xi_k}{(1-\alpha)N}
\end{equation}

Note that the first term in the right-hand side of equation appears because of inequality greater than equal to $\hat{\alpha}$. If (1 \(-\) $\hat{\alpha}$) \* 100\% of the worst drawdowns can be counted precisely, then  and the first term in the right-hand side of equation disappears. Equation follows from the framework of the CVaR methodology


\section{Usage}

In this example we use edhec database, to compute true Hedge Fund Returns.

\begin{Schunk}
\begin{Sinput}
> library(PerformanceAnalytics)
> data(edhec)
> CDrawdown(edhec)
\end{Sinput}
\begin{Soutput}
                             Convertible Arbitrage CTA Global
Conditional Drawdown at Risk            -0.2524256 -0.1183364
                             Distressed Securities Emerging Markets
Conditional Drawdown at Risk            -0.1423125       -0.4464759
                             Equity Market Neutral Event Driven
Conditional Drawdown at Risk            -0.0578064   -0.1582837
                             Fixed Income Arbitrage Global Macro
Conditional Drawdown at Risk             -0.1503749  -0.07968906
                             Long/Short Equity Merger Arbitrage Relative Value
Conditional Drawdown at Risk        -0.1995105      -0.08693944     -0.1079869
                             Short Selling Funds of Funds
Conditional Drawdown at Risk    -0.8982062     -0.1600725
\end{Soutput}
\end{Schunk}



\end{document}
