%% no need for  \DeclareGraphicsExtensions{.pdf,.eps}

\documentclass[12pt,letterpaper,english]{article}
\usepackage{times}
\usepackage[T1]{fontenc}
\IfFileExists{url.sty}{\usepackage{url}}
                      {\newcommand{\url}{\texttt}}

\usepackage{babel}
%\usepackage{noweb}
\usepackage{Rd}

\usepackage{Sweave}

%\VignetteIndexEntry{Performance Attribution from Bacon}
%\VignetteDepends{PerformanceAnalytics}
%\VignetteKeywords{returns, performance, risk, benchmark, portfolio}
%\VignettePackage{PerformanceAnalytics}

%\documentclass[a4paper]{article}
%\usepackage[noae]{Sweave}
%\usepackage{ucs}
%\usepackage[utf8x]{inputenc}
%\usepackage{amsmath, amsthm, latexsym}
%\usepackage[top=3cm, bottom=3cm, left=2.5cm]{geometry}
%\usepackage{graphicx}
%\usepackage{graphicx, verbatim}
%\usepackage{ucs}
%\usepackage[utf8x]{inputenc}
%\usepackage{amsmath, amsthm, latexsym}
%\usepackage{graphicx}

\title{Normalized Calmar and Sterling Ratio}
\author{R Project for Statistical Computing}

\begin{document}
\Sconcordance{concordance:NormCalmar.tex:NormCalmar.rnw:%
1 47 1 1 5 1 4 42 1 1 2 1 0 2 1 14 0 1 1 19 0 1 2 3 1}


\maketitle


\begin{abstract}
 Both the Calmar and the Sterling ratio are the ratio of annualized returnmover the absolute value of the maximum drawdown of an investment. The Sterling ratio adds an excess risk measure to the maximum drawdown,  traditionally and defaulting to 10\%.It is also traditional to use a three year return series for these
 calculations, although the functions included here make no effort to
 determine the length of your series. However, Malik Magdon-Ismail devised a scaling law in which can be used to compare Calmar/Sterling ratio's with different 
$\mu$ ,$\sigma$ and T.
\end{abstract}



\section{Background}
Given a sample of historical returns \((R_1,R_2, . . .,R_T)\),the Calmar and Sterling Ratio's are defined as :

%Let $X \sim N(0,1)$ and $Y \sim \textrm{Exponential}(\mu)$.  Let
%$Z = \sin(X)$. $\sqrt{X}$.
  
%$\hat{\mu}$ = $\displaystyle\frac{22}{7}$
%e^{2 \mu} = 1
%\begin{equation}
%\left(\sum_{t=1}^{T} R_t/T\right) = \hat{\mu} \\
%\end{equation}
\begin{equation}
 Calmar Ratio  =  \frac{Return [0,T]}{max Drawdown  [0,T]} \\
\end{equation}

\begin{equation}
 Sterling Ratio  =  \frac{Return [0,T]}{max Drawdown  [0,T] - 10\%} \\
\end{equation}

\section{Scaling Law}
Malik Magdon-Ismail  impmemented a sclaing law for different $\mu$ ,$\sigma$ and T.Defined as :


\begin{equation}
Calmar_{\tau}  =  \gamma(_{\tau , Sharpe_{1}})Calmar_{T_{1}}  \\
\end{equation}

Where : 
  \begin{equation}
\gamma(_{\tau , Sharpe_{1}})  =  \frac{\frac{Q_p(T_1/2Sharpe^2_{1})}{T_1}}{\frac{Q_p(T_2/2Sharpe^2_{1})}{\tau}} \\
\end{equation}

 And , when T tends to  Infinity
\begin{equation}
Q_p(T/2Sharpe^2)  =  .63519 + log (Sharpe)  + 0.5 log T\\
\end{equation}

Same methodolgy goes to Sterling Ratio.
\section{Usage}

In this example we use edhec database, to compute Calmar and Sterling Ratio.

\begin{Schunk}
\begin{Sinput}
> library(PerformanceAnalytics)
> data(edhec)
> CalmarRatio.Normalized(edhec,1)
\end{Sinput}
\begin{Soutput}
                        Convertible Arbitrage CTA Global Distressed Securities
Normalized Calmar Ratio            0.05538467  0.1779411            0.07219164
                        Emerging Markets Equity Market Neutral Event Driven
Normalized Calmar Ratio        0.1118862            0.09525316   0.08067917
                        Fixed Income Arbitrage Global Macro Long/Short Equity
Normalized Calmar Ratio             0.06372551    0.1977305        0.08391112
                        Merger Arbitrage Relative Value Short Selling
Normalized Calmar Ratio        0.2184794      0.0813596 -0.0006817146
                        Funds of Funds
Normalized Calmar Ratio     0.07172177
\end{Soutput}
\begin{Sinput}
> SterlingRatio.Normalized(edhec,1)
\end{Sinput}
\begin{Soutput}
                                         Convertible Arbitrage CTA Global
Normalized Sterling Ratio (Excess = 10%)             0.0412807 0.09585286
                                         Distressed Securities Emerging Markets
Normalized Sterling Ratio (Excess = 10%)            0.05026439       0.08755194
                                         Equity Market Neutral Event Driven
Normalized Sterling Ratio (Excess = 10%)            0.05007166   0.05385919
                                         Fixed Income Arbitrage Global Macro
Normalized Sterling Ratio (Excess = 10%)             0.04086785   0.08740785
                                         Long/Short Equity Merger Arbitrage
Normalized Sterling Ratio (Excess = 10%)        0.05754033        0.0787349
                                         Relative Value Short Selling
Normalized Sterling Ratio (Excess = 10%)     0.04999597 -0.0005672599
                                         Funds of Funds
Normalized Sterling Ratio (Excess = 10%)     0.04827673
\end{Soutput}
\end{Schunk}

We can see as we shrunk the period the Ratio's decrease because the Max Drawdown does not change much over reduction of time period, but returns are approximately scaled according to  the time length. 

\end{document}
