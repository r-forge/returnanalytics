%% no need for  \DeclareGraphicsExtensions{.pdf,.eps}

\documentclass[12pt,letterpaper,english]{article}
\usepackage{times}
\usepackage[T1]{fontenc}
\IfFileExists{url.sty}{\usepackage{url}}
                      {\newcommand{\url}{\texttt}}

\usepackage{babel}
%\usepackage{noweb}
\usepackage{Rd}

\usepackage{Sweave}

%\VignetteIndexEntry{Performance Attribution from Bacon}
%\VignetteDepends{PerformanceAnalytics}
%\VignetteKeywords{returns, performance, risk, benchmark, portfolio}
%\VignettePackage{PerformanceAnalytics}

%\documentclass[a4paper]{article}
%\usepackage[noae]{Sweave}
%\usepackage{ucs}
%\usepackage[utf8x]{inputenc}
%\usepackage{amsmath, amsthm, latexsym}
%\usepackage[top=3cm, bottom=3cm, left=2.5cm]{geometry}
%\usepackage{graphicx}
%\usepackage{graphicx, verbatim}
%\usepackage{ucs}
%\usepackage[utf8x]{inputenc}
%\usepackage{amsmath, amsthm, latexsym}
%\usepackage{graphicx}

\title{Maximum Drawdown of a Brownian Motion}
\author{R Project for Statistical Computing}

\begin{document}
\Sconcordance{concordance:ExpectedMaxDDGMB.tex:ExpectedMaxDDGMB.Rnw:%
1 45 1 1 5 1 4 9 1 1 2 1 0 2 1 25 0 1 2 3 1}


\maketitle


\begin{abstract}
if $\hat{X}$(t) is a random process on [0,T], the maximum drawdown in defined as the largest drop from a peak to a bottom.This paper investigates the behavior of this statistic for a Brownian motion with drift. In particular, it gives an  $\infty$ series representation of its distribution, and consider its expected value. When the drift is zero, it gives an analytic expression for the expected value, and for non-zero drift, it gives an  $\infty$ series representation.For all cases, we compute the limiting  T tends to $\infty$ behavior, which can be
logarithmic ($\mu$ greater than 0), square root ( $\mu$ equal to 0), or linear ($\mu$ less than 0).  
\end{abstract}



\section{Background}

The maximum drawdown is a commonly used in finance as a measure of risk for a stock that follows a particular random process. Here we consider the maximum drawdown of a Brownian motion.


\section{Usage}

In this example we use edhec database, to compute true Hedge Fund Returns.

\begin{Schunk}
\begin{Sinput}
> library(PerformanceAnalytics)
> data(edhec)
> table.EMaxDDGBM(edhec)
\end{Sinput}
\begin{Soutput}
                       Convertible Arbitrage CTA Global Distressed Securities
Annual Returns in %                   7.7020     7.6711                9.7510
Std Devetions in %                    2.0047     2.5131                1.8348
Expected Drawdown in %                1.7083     2.5086                1.2412
                       Emerging Markets Equity Market Neutral Event Driven
Annual Returns in %              9.3612                7.3936       9.3190
Std Devetions in %               3.8571                0.9006       1.8350
Expected Drawdown in %           4.4855                0.4418       1.2828
                       Fixed Income Arbitrage Global Macro Long/Short Equity
Annual Returns in %                    5.0675       9.4208            9.4015
Std Devetions in %                     1.4171       1.7020            2.2174
Expected Drawdown in %                 1.2693       1.1179            1.7631
                       Merger Arbitrage Relative Value Short Selling
Annual Returns in %              8.3721         8.2317        3.2654
Std Devetions in %               1.1168         1.3195        5.5099
Expected Drawdown in %           0.5866         0.7941       14.0675
                       Funds of Funds
Annual Returns in %            7.1270
Std Devetions in %             1.8212
Expected Drawdown in %         1.5320
\end{Soutput}
\end{Schunk}



\end{document}
