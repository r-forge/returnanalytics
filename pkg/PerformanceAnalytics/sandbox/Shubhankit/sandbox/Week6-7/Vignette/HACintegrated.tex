\documentclass[nojss]{jss}
\usepackage{thumbpdf}
%% need no \usepackage{Sweave}

\author{Achim Zeileis\\Wirtschaftsuniversit\"at Wien}
\Plainauthor{Achim Zeileis}

\title{Econometric Computing with HC and HAC Covariance Matrix Estimators}

\Keywords{covariance matrix estimators, heteroskedasticity, autocorrelation,
  estimating functions, econometric computing, \proglang{R}}
\Plainkeywords{covariance matrix estimators, heteroskedasticity, autocorrelation,
  estimating functions, econometric computing, R}

\Abstract{
This introduction to the \proglang{R} package \pkg{sandwich} is a (slightly)
modified version of \cite{hac:Zeileis:2004a}, published in the
\emph{Journal of Statistical Software}. A follow-up paper on object object-oriented
computation of sandwich estimators is available in \citep{hac:Zeileis:2006}.

Data described by econometric models typically contains autocorrelation 
and/or heteroskedasticity of unknown form and for inference in such models
it is essential to use covariance matrix estimators that can consistently
estimate the covariance of the model parameters. Hence, suitable heteroskedasticity-consistent
(HC) and heteroskedasticity and autocorrelation consistent (HAC) estimators
have been receiving attention in the econometric literature over the last
20 years. To apply these estimators in practice, an implementation is needed
that preferably translates the conceptual properties of the underlying
theoretical frameworks into computational tools. In this paper, such an implementation
in the package \pkg{sandwich} in the \proglang{R} system for statistical
computing is described and it is shown how the suggested functions
provide reusable components that build on readily existing functionality
and how they can be integrated easily into new inferential procedures or applications.
The toolbox contained in \pkg{sandwich}
is extremely flexible and comprehensive, including specific functions for the most important HC
and HAC estimators from the econometric literature.
Several real-world data sets are used to illustrate how the functionality
can be integrated into applications.
}



\end{document}
